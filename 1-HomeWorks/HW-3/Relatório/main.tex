\documentclass[a4paper,11pt]{article}


\usepackage{preambulo} 
 
\renewcommand{\lstlistingname}{Listado}
\lstset{
    backgroundcolor=\color[rgb]{0.86,0.88,0.93},
    language=R, keywordstyle=\color[rgb]{0,0,1},
    basicstyle=\footnotesize \ttfamily,breaklines=true,
    escapeinside={\%*}{*)}
}
\usepackage{footmisc} \renewcommand{\labelitemi}{$\circ$}
\usepackage{enumitem} \setlist[itemize]{leftmargin=*}

\usepackage{scrextend}
\deffootnote[1em]{1em}{1em}{\textsuperscript{\thefootnotemark}\,}

%%%%%%%%%% Document starts here %%%%%%%%%%%

\begin{document}
%%%%%%%%%% Title %%%%%%%%%%%
\begin{figure}[!h] \includegraphics [scale=0.3] {Imagens/extras/Course-logo} \end{figure}

\begin{spacing}{1.5}
{\Large\sc \noindent Homework III} \\

{\large\sc \noindent Nome completo: Lucas de Oliveira Sobral}\\
%{\large\sc \noindent Nome completo:}\\
{\large\sc \noindent Numero de matricula: 556944}\\
%{\large\sc \noindent Numero de matricula: }\\
{\large\sc \noindent Nome completo: Álvaro José Passos de Freitas Neto}\\
%{\large\sc \noindent Nome completo:}\\
{\large\sc \noindent Numero de matricula: 567593}
%{\large\sc \noindent Numero de matricula: }
\end{spacing}

\vskip1cm

%%%%%%%%%% Content starts here %%%%%%%%%%%
\section{Questão}  \label{sec:q1}
Assume-se que o tempo de vida $X$ (medido em anos) de um computador segue uma distribuição exponencial com parâmetro desconhecido $\lambda > 0$. Uma amostra aleatória dos tempos de vida dos computadores é apresentada na Tabela \ref{tab:tabela1}. Os dados são fictícios e são utilizados apenas para fins ilustrativos.


\begin{table}[h!]
\centering
\label{tab:tabela1}
    \begin{tabular}{cccccccccc}
    0.99 & 2.31 & 10.85 & 6.15 & 10.81 & 3.72 & 5.75 & 4.15 & 9.27 & 7.84 \\
    2.31 & 10.85 & 6.15 & 1.81 & 3.72 & 5.75 & 10.40 & 10.04 & 4.15 & 9.27 \\
    \end{tabular}
\caption{Dados utilizados na questão \ref{sec:q1}: Tempo de vida (em anos) dos computadores.}
\end{table}



\begin{enumerate}[leftmargin=*]
\item Escreva a função densidade de probabilidade da distribuição exponencial com parâmetro $\lambda$.

\item Dada uma amostra aleatória $X1, X2, . . . , Xn$:
\begin{enumerate}
    \item Escreva a função de verossimilhança $L(\lambda)$.
    \item Derive a correspondente função log-verossimilhança $\ell(\lambda)$.
    \item Determine o estimador de máxima verossimilhança (MLE, do inglês) $\hat{\lambda}$ de $\lambda$.
\end{enumerate}



\item Utilizando os dados fornecidos na Tabela \ref{sec:q1}, calcule o valor numérico do MLE $\hat{\lambda}$.

\item Construa o gráfico da função log-verossimilhança $\ell(\lambda)$ com base nos dados observados,
considerando um intervalo adequado de valores para $\lambda$. Indique claramente no gráfico
o valor do estimador de máxima verossimilhança $\hat{\lambda}$. 
    
\item Utilizando o parâmetro estimado $\hat{\lambda}$:
    \begin{enumerate}
        \item Calcule o tempo médio de vida estimado de um computador.
        \item Calcule a probabilidade de que um computador funcione por mais de 5 anos.
    \end{enumerate}

\item A distribuição exponencial possui a $\textit{propriedade da falta de memória}$, o que significa que a probabilidade de falha no futuro não depende do tempo que o computador já esteve em funcionamento.
    \begin{enumerate}
        \item Explique essa propriedade com suas próprias palavras.
        \item Discuta brevemente se essa suposição parece razoável para modelar o tempo de vida de computadores.
    \end{enumerate}

\end{enumerate} 


\subsection*{Solução da questão} 			

\subsubsection*{Descrição da atividade}
.....

\subsubsection*{.......:} 

\begin{itemize}
\item[]

\text{\large O que é uma variável aleatória (\(X\))?}
    \item É uma função que atribui um valor numérico a cada resultado de um experimento aleatório $X(s)$.

\begin{comment}
\begin{figure}[H] 
    \centering 
\includegraphics[width=0.4\textwidth]{Imagens/Graficos/funçãoX.png} 
\end{figure}

\end{comment}
    
\end{itemize}

\subsubsection*{Respostas dos itens da questão 1:} 

\begin{description}

\item [1.1] \textbf{ Resposta}: \\
.....

\item [1.2] \textbf{ Resposta}: \\
....

\item [1.3] \textbf{ Resposta}: \\
....

\item [1.4] \textbf{ Resposta}: \\
....

\item [1.5] \textbf{ Resposta}: \\
....

\item [1.6] \textbf{ Resposta}: \\
....

\end{description}

\section{Questão} \label{sec:q2}
O conjunto de dados de penguins, na biblioteca palmerpenguins3 do R, contém medidaspara as três espécies de pinguins (figura \ref{img:pinguins}): ilha no arquipélago Palmer na Antártica,tamanho (comprimento da nadadeira, massa corporal, dimensões do bico) e sexo. Importeo conjunto de dados4 e familiarize com ele.

\begin{figure}[H] 
    \centering 
\includegraphics[width=0.9\textwidth]{Imagens/Graficos/pinguins.png} 
\caption{ Espécies e características dos pinguins na questão \ref{sec:q2}}
    \label{img:pinguins}
\end{figure}


\begin{enumerate}[leftmargin=*]
\item Considere a massa corporal \texttt{body\_mass}
 em gramas como variável independente, x, e o comprimento do bico \texttt{bill\_length}em milímetros como variável dependente y. Construa um gráfico de dispersão entre x e y. Com base no gráfico, comente se uma
relação linear entre as variáveis parece plausível.

\item \label{it:item 2} Defina os parâmetros da reta de regressão com o método dos mínimos quadrados e verifique os resultados obtidos com o comando $lm()$ no R. Adicione a reta de regressão no gráfico de dispersão.

\item Calcule os resíduos da regressão e apresente uma representação gráfica dos mesmos. Em seguida, calcule a raiz do erro quadrático médio (RMSE, do inglês) e o coeficiente de determinação $R^2$. Comente sobre os resultados obtidos.

\item O conjunto de dados não apresenta outliers evidentes. Modifique esse conjunto introduzindo artificialmente uma observação extrema, seja por meio de um aumento ou
de uma redução substancial no valor da massa corporal ou do comprimento do bico
de um dos pinguins. Em seguida, ajuste um modelo de regressão linear utilizando
o conjunto de dados modificado. Compare os coeficientes estimados da regressão, as
retas ajustadas e os valores do $RMSE$ e do $R^2$
com aqueles obtidos no item \ref{it:item 2}. Por
fim, discuta a influência da observação artificialmente introduzida sobre os resultados
da regressão

\end{enumerate}

\subsection*{Solução da questão} 		

\subsubsection*{Descrição da atividade}
......

\subsubsection*{Mais distribuições especiais discretas } 

\begin{itemize}
\item[]

\text{\large ......}
    \item .......

    
\end{itemize}


\subsubsection*{Respostas dos itens da questão 2:} 

\begin{description}[leftmargin=*]

\item [2.1] \textbf{ Resposta}: \\
.....

\item [2.2] \textbf{ Resposta}: \\
....

\item [2.3] \textbf{ Resposta}: \\
....

\item [2.4] \textbf{ Resposta}: \\
....

\end{description}

\newpage
\begin{appendices}


\end{appendices}

\end{document}
