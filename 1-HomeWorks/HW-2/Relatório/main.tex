\documentclass[a4paper,11pt]{article}


\usepackage{preambulo} 
 
\renewcommand{\lstlistingname}{Listado}
\lstset{
    backgroundcolor=\color[rgb]{0.86,0.88,0.93},
    language=R, keywordstyle=\color[rgb]{0,0,1},
    basicstyle=\footnotesize \ttfamily,breaklines=true,
    escapeinside={\%*}{*)}
}
\usepackage{footmisc} \renewcommand{\labelitemi}{$\circ$}
\usepackage{enumitem} \setlist[itemize]{leftmargin=*}

\usepackage{scrextend}
\deffootnote[1em]{1em}{1em}{\textsuperscript{\thefootnotemark}\,}

%%%%%%%%%% Document starts here %%%%%%%%%%%

\begin{document}
%%%%%%%%%% Title %%%%%%%%%%%
\begin{figure}[!h] \includegraphics [scale=0.3] {Imagens/extras/Course-logo} \end{figure}

\begin{spacing}{1.5}
{\Large\sc \noindent Homework II} \\

{\large\sc \noindent Nome completo: Lucas de Oliveira Sobral}\\
%{\large\sc \noindent Nome completo:}\\
{\large\sc \noindent Numero de matricula: 556944}\\
%{\large\sc \noindent Numero de matricula: }\\
{\large\sc \noindent Nome completo: Álvaro José Passos de Freitas Neto}\\
%{\large\sc \noindent Nome completo:}\\
{\large\sc \noindent Numero de matricula: 567593}
%{\large\sc \noindent Numero de matricula: }
\end{spacing}

\vskip1cm

%%%%%%%%%% Content starts here %%%%%%%%%%%
\section{Questão}  \label{sec:q1}
Em um restaurante muito frequentado, aproximadamente 70\% dos clientes pedem uma sobremesa após o prato principal. Seja X a variável aleatória que representa o número de
clientes que pedem sobremesa em uma amostra aleatória de n = 50 clientes.


\begin{enumerate}[leftmargin=*]
\item Determine a função de distribuição de $X$.

\item Construa os gráficos da função massa de probabilidade (PMF) e da função distribuição
acumulada (CDF) de $X$.

\item Calcule o valor esperado, a variância e o desvio padrão de $X$.

\item Calcule a probabilidade de:

    \begin{enumerate}
        \item $P(X \ge 20)$.
        \item $P(30 < X < 43)$.
        \item $P(X=31)$.
    \end{enumerate}
    
\item Suponha que o restaurante estoque sobremesas com base na demanda esperada. Como
o uso da distribuição de X poderia ajudar a reduzir desperdício e evitar falta de
produtos?

\item Como mudanças em $p$ (por exemplo, sobremesa se torna mais popular, $p = 0.8$) ou em
$n$ (número de clientes) afetariam a forma e as probabilidades de X?

\end{enumerate} 


\subsection*{Solução da questão} 			

\subsubsection*{Descrição da atividade}
......

\subsubsection*{Variáveis aleatórias:} 

\begin{itemize}
\item[]

\text{\large O que é uma variável aleatória (\(X\))?}
    \item É uma função que atribui um valor numérico a cada resultado de um experimento aleatório $X(s)$.


\begin{figure}[H] 
    \centering 
\includegraphics[width=0.4\textwidth]{Imagens/Graficos/funçãoX.png} 
\end{figure}

    
\text{\large Tipos de variáveis aleatórias}
    \item Variáveis aleatórias discretas: É uma variável aleatória que pode assumir um número contável de valores, seja ele finito ou infinito de valores e explicaremos melhor nessa questão.
    \item Variáveis aleatórias contínuas: São variáveis que podem assumir qualquer número dentro de um intervalo contínuo e falaremos melhor depois. 
\end{itemize}

\begin{itemize}
\item[]

\text{\large Valor médio esperado $E(X)$}
    \item Uma das maiores diferenças entre as variáveis aleatórias da probabilidade e variáveis da estatística descritiva se dá pela incapacidade de afirmar com 100\% de certeza os valores da amostragem em um experimento, por isso temos uma constante determinística do modelo, chamado de "esperança" \space média de valores. Entre variáveis discretas e contínuas há uma diferença sutil no cálculo da esperança, que será visto mais à frente.

\text{\large Variância de uma variável aleatória $Var$($\sigma^2$ ou $S^2$ )}
    \item É o valor de o quão afastados os dados estão da média esperada $E(X)$, se a variância tiver um baixo valor então os dados estão concentrados no ponto médio, se estiver com valor alto indica que os dados estão mais espalhados. 
    \item Para variáveis discretas e contínuas a fórmula formal é muito complicada de calcular por isso usamos uma técnica inconsciente, chamada de LOTUS para simplificar os cálculos, apresentada a seguir:

\text{\large LOTUS Lei do estatístico inconsciente}
    \item É a regra geral para calcular a esperança média de qualquer função de uma variável aleatória, sem precisar descobrir a distribuição dessa nova função, ela diz que aplicar os valores $x_i$ na nova função $g(x)$ e multiplicar pela probabilidade de $x_i$ .

    \item Fórmula:

    \[E[g(x)] = \sum g(x)*P(X=x_i)\]

\text{\large Distribuição de uma variável aleatória}
    \item É uma forma de descrição matemática que resulta em uma visualização de como os valores estão se distribuindo em um gráfico que a soma de suas probabilidades resulta em 1 e não podem ter probabilidade negativa. Suas principais funções são:
    
    \item Função massa de probabilidade FMP ou $(PMF)$ para discretas
    \item Função densidade de probabilidade FDP ou $(PDF)$ para contínuas
    \item Função de distribuição acumulada FDA ou $(CDF)$ para discretas e contínuas

\text{\large Distribuições especiais}
    \item São modelos matemáticos para situações que acontecem na vida real, algumas delas são:
    \begin{table}[h]
        \centering
        
        \label{tab:distribuicoes}
        \begin{tabular}{|c|c|c|c|}
        \hline
        \textbf{Tipo} & \textbf{Distribuição} & \textbf{Notação} & \textbf{Parâmetros} \\ \hline
        \multirow{4}{*}{Discretas} & Bernoulli & $X \sim Bern(p)$ & $p$ (prob. sucesso) \\ \cline{2-4} 
         & Binomial & $X \sim Bin(n, p)$ & $n$ (tentativas), $p$ \\ \cline{2-4} 
         & Geométrica & $X \sim Geom(p)$ & $p$ (prob. sucesso) \\ \cline{2-4} 
         & Poisson & $X \sim Pois(\lambda)$ & $\lambda$ (taxa média) \\ \hline
        \multirow{3}{*}{Contínuas} & Uniforme & $X \sim U(a, b)$ & $a$ (min), $b$ (max) \\ \cline{2-4} 
         & Normal & $X \sim N(\mu, \sigma^2)$ & $\mu$ (média), $\sigma^2$ (variância) \\ \cline{2-4} 
         & Exponencial & $X \sim Exp(\lambda)$ & $\lambda$ (taxa) \\ \hline
        \end{tabular}
        \caption{Principais Distribuições Especiais}
\end{table}


\end{itemize}

\subsubsection*{Variáveis aleatórias discretas:} 

\begin{itemize}
\item[]

\text{\large O que são?}
    \item É a variável aleatória que pode assumir um número contável de valores, seja ele finito ou infinito.


\text{\large Função massa de probabilidade FMP ou (\(PMF\))}
    \item É a função que nos dá a probabilidade exata de a variável assumir um valor específico, exemplo de PMF com $Rx=({0,1,2})$ e $Px=({1/4,1/2,1/4})$.

\begin{figure}[H] 
    \centering 
\includegraphics[width=0.8\textwidth]{Imagens/Graficos/exemploPMF.png} 
\end{figure}



\text{\large Função de distribuição acumulada discreta FDA ou (\(CDF\))}
    \item dsfadasdasdsad


\text{\large Valor médio esperado $E(X)$ de uma variável discreta}
    \item Para variáveis discretas o cálculo é feito pela média ponderada das amostradas $x_i$ por suas probabilidades 
    $P_X(x_i)$ ou $P(X=x_i)$.

    \item Fórmula:

    \[E(X) = \sum_i x_iP(X=x_i)\]

\text{\large Variância de uma variável aleatória discreta $Var(X)$}
    \item A variância de uma variável aleatória discreta possui a seguinte fórmula formal:
        $$ Var(X) = E[(X - \mu)^2] = \sum_x (x - \mu)^2 \cdot P(X=x) $$
    \item Mas utilizando a técnica de LOTUS podemos obter:
        \[Var(X) = E(X^2) - [E(X)]^2\]
        \[Var(X) = [\sum_i (x_i)^2 \cdot P(X=x_i)] - [E(x)]^2\]
\end{itemize}




\begin{description}

\item \textbf{[1.1] Resposta}:


\end{description}

\section{Questão} \label{sec:q2}
Um site realiza uma pesquisa online e oferece uma recompensa a um usuário selecionado aleatoriamente que responde a uma série de perguntas. Cada um dos 10 milhões de visitantes diários tem, independentemente, probabilidade $p = 10^7$ de ganhar a recompensa.



\begin{enumerate}[leftmargin=*]
\item Encontre uma aproximação simples e adequada para a função de massa de probabilidade (PMF) do número de vencedores em um dia, $X$. Justifique claramente se
essa aproximação é apropriada para os valores dados de n e $p$.

\item  Calcule o valor esperado, $E[X]$, e a variância $Var(X)$, usando tanto a distribuição exata
quanto a aproximada. Comente sobre a semelhança entre os resultados.

\item Suponha que você ganhe a recompensa, mas que possa haver outros vencedores. Seja
$W \thicksim Pois(1)$ o número de vencedores além de você. Se houver vários vencedores, o
prêmio é sorteado aleatoriamente entre todos eles. Encontre a probabilidade de que
você realmente receba o prêmio.

\item Gere um grande número de simulações diárias para o número de vencedores. Crie
uma comparação visual entre os resultados empíricos e a aproximação considerada
no item ??. Descreva brevemente o que a visualização indica sobre a qualidade da
aproximação.

\end{enumerate}

\subsection*{Solução da questão} 		

\subsubsection*{Descrição da atividade}
.....

\subsubsection*{Medidas de dispersão - Base Teórica:} 

\begin{itemize}
\item[]

\text{\large Coeficiente de variação de Pearson ($r_{xy)}$}
    \item É uma medida estatística que descreve o grau e a direção da associação linear entre duas variáveis quantitativas (X e Y).

    \[r_{xy} = \frac{\sum{(x_i - \overline{x})(y_i-\overline{y})}}{\sqrt{\sum (x_i - \overline{x})^2 \sum(y_i-\overline{y})^2}}\]

    O resultado estará em um intervalo de [-1,1], 

\end{itemize}



\begin{description}[leftmargin=*]

\item[2.1] \textbf{Resposta}:

\end{description}

\section{Questão} \label{sec:q3}
Você é responsável por monitorar a temperatura de uma CPU multicore em uma unidade de processamento embarcada. Sob carga normal, a temperatura da CPU apresenta flutuações devido a mudanças na carga de trabalho, nas condições ambientais e na eficiência do sistema de resfriamento. Testes mostram que a temperatura em regime estacionário da CPU segue uma distribuição normal com temperatura média $\mu = 62$ ºC e desvio padrão $\sigma = 3,5$ ºC. Sua tarefa é simular medições de temperatura da CPU e analisar suas propriedades estatísticas.

\begin{enumerate}[leftmargin=*]
\item Crie uma função que gere valores com distribuição normal usando a transformação
de Box-Muller, a partir de entradas aleatórias uniformes. Especificamente:
    \begin{enumerate}
        \item Gere duas variáveis aleatórias uniformes independentes: $U_1,U_2 \thicksim Unif(0,1)$
        \item Calcule dois valores normais padrão usando as fórmulas de Box-Muller:
        \[Z_1 = \sqrt{-2ln(U_1)}cos(2\pi U_2), Z_2 = \sqrt{-2ln(U_1}\sin(2\pi U_2)\]
        $Z_1$ e $Z_2$ são variáveis aleatórias independentes com distribuição normal padrão.
Concatene-as para formar um vetor Z de valores normais padrão.
        \item Converta cada valor normal padrão para a distribuição de temperatura da CPU:
        \[T = 62+3.5Z\]
    \end{enumerate}

\item Use seu gerador de números aleatórios para gerar 1.000 medições de temperatura da
CPU.
Gere mais 1.000 valores de temperatura utilizando o gerador de números aleatórios
normal embutido do R, com a mesma média e desvio-padrão.

\item Para ambos os conjuntos de dados simulados, calcule:

    \begin{enumerate}
        \item Média amostral.
        \item Desvio-padrão amostral.
        \item Temperatura mínima e máxima observada.
        \item Probabilidade empírica e teórica $P(T>68)$
        \item Probabilidade empírica e teórica $P(60<T<65)$
        \item Probabilidade teórica $P(T>75$.
        Algum dos conjuntos de dados simulados (1.000 amostras) contém valores acima
de 75◦C? Caso não, explique por que eventos raros requerem tamanhos de amostra grandes para serem observados.

        \end{enumerate}

\item Visualize os resultados criando
    \begin{enumerate}
        \item Um histograma das temperaturas simuladas da CPU (pode plotar os dois conjuntos de dados separadamente ou sobrepostos).
        \item A função densidade de probabilidade (PDF) normal teórica (média 62 ◦C, desvio
padrão 3,5 ºC) sobreposta ao histograma.
    \end{enumerate}

\item Discuta seus resultados respondendo às seguintes perguntas: As distribuições empíricas da temperatura da CPU se assemelham à curva normal teórica? Quão próximas estão a média amostral e o desvio-padrão amostral dos valores esperados 62 ◦C e 3, 5
◦C?
Há diferenças perceptíveis entre o conjunto de dados gerado com seu RNG manual e
o produzido pelo RNG embutido do R? Como essa simulação pode ajudar na avaliação de estratégias de resfriamento ou de escalonamento dinâmico de clock? Por que
geradores de números aleatórios uniformes são a base dos sistemas de RNG?

\end{enumerate}

\subsection*{Solução da questão} 					

\subsubsection*{Descrição da atividade}
....


\subsubsection*{Base Teórica Questão 3:} 

\begin{itemize}
\item[]

\subsubsection*{Tipos de dados - base teórica} 

    
\end{itemize}


\begin{description}[leftmargin=*] 

\item[3.1] \textbf{ Resposta}: \\

\end{description}

\newpage
\begin{appendices}


\end{appendices}

\end{document}
